\documentclass{Revue-economique} % Classe de document pour la revue économique

\addbibresource[label=ExempleBib]{ExempleBib.bib} % Fichier biblio pour l'exemple. Le label sert à insérer les biblios plus loin dans le document.

\begin{document} % Contenu du document
\begin{Article} [% Ceci correspond à la page de garde du document
    Titre={<Titre de l'article dans la langue principale>},
    Auteur={Auteur 1\thanks{Affiliation et correspondance}, \par
            Auteur 2\thanks{Affiliation et correspondance}, \par
            Auteur 3\thanks{Affiliation et correspondance}, \par
            ...}]

\begin{resume}
    % Résumé dans la langue principale
\end{resume}

\titrearticleENG{<Titre de l'article en anglais ou dans la langue secondaire>}

\begin{resumeENG}
    % Résumé en anglais ou dans la langue secondaire
\end{resumeENG}

\motscles{<Mots-clés>} % Mots-clés en français
\keywords{<Keywords>} % Mots-clés en anglais

\jelcode{<JEL>} % Codes JEL

\begin{refsection}[ExempleBib] % Corps de texte
    
\section{Introduction}

Paragraphe introductif 
\textcite{smith2023economic}, % Pour obtenir : Smith [2023]
\parencite{smith2023economic}. % Pour obtenir : (Smith [2023])

\begin{appendices}
    % Contenu de l'annexe
\end{appendices}

\printbibliography % Bibliographie
\end{refsection}

\end{Article}

\end{document}
